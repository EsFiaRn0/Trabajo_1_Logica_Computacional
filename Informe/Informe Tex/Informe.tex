\documentclass{report}

\usepackage{epsfig}
\usepackage{graphicx}
\usepackage{array}
\usepackage{geometry}
\usepackage{longtable}
\usepackage{amsmath}
\usepackage{amssymb}
\usepackage{tikz}

\renewcommand*\thesection{\arabic{section}}

\newcommand \minitab{\hspace*{15 pt}}
\newcommand{\floor}[1]{\left\lfloor #1 \right\rfloor}

\hyphenation{ins-truc-ciones}

\begin{document}
	\begin{titlepage}
		\begin{center}
			\includegraphics[width=0.5\textwidth]{../Imagenes/LOGO-USACH_COLOR}
		\end{center}
		\begin{center}
			{\bf Departamento de Matem\'atica y Ciencia de la Computaci\'on}
		\end{center}
		\vspace{3cm}
		\begin{center}
			{\Large \bf Tarea 1 - SAT Lineal}
			~ \\ 
			~ \\ 
			~ \\
			~ \\
			\begin{tabular}{c c c}
				Miguel Olivares Morales & ~~~~~~~ & Benjam\'in Riveros Landeros \\
				miguel.olivares@usach.cl & & benjamin.riveros.l@usach.cl \\
			\end{tabular}
			~ \\ 
			~ \\ 
			~ \\
			~ \\
			\begin{tabular}{c c c}
				L\'ogica Computacional - 22625 & ~~~~~~~~~~~~~~~~~ & Semestre Oto\~no 2025 \\
				Licenciatura en Ciencia de la Computaci\'on & &  \\
			\end{tabular}
		\end{center}
	\end{titlepage}
	
	\section{Introducci\'on}
	El problema de determinar si las variables de una f\'ormula booleana pueden ser reemplazadas con valores \textbf{T} o \textbf{F} de tal forma que la f\'ormula de como resultado \textbf{T} se denomina problema de satisfacibilidad booleana o SAT. Si al evaluar la f\'ormula esta da como resultado \textbf{T}, entonces se dice que es satisfactoria.
	\section{Procedimiento}
	Las f\'ormulas que ser\'an analizadas primero tendr\'an que ser codificadas seg\'un la siguiente gramatica:
	\[ \phi ::= p \hspace{1mm}|\hspace{1mm} (\neg \phi)  \hspace{1mm}|\hspace{1mm} (\phi \land \phi) \]
	Para esto usamos el siguiente esquema de traducci\'on: \\\\
	\begin{minipage}[t]{0.45\textwidth}
		$ T(p)=p $ \\\\
		$ T(\phi_1 \land \phi_2)=T(\phi_1) \land T(\phi_2) $ \\\\
		$ T(\phi_1 \rightarrow \phi_2) = \neg(T(\phi_1) \land \neg T(\phi_2)) $
	\end{minipage}
	\hfill
	\begin{minipage}[t]{0.45\textwidth}
		$ T(\neg \phi) = \neg T(\phi) $ \\\\
		$ T(\phi_1 \lor \phi_2) = \neg(\neg T(\phi_1) \land \neg T(\phi_2)) $
	\end{minipage} \vspace*{5mm} \\ 
	Esto quiere decir que se analizar\'an f\'ormulas compuestas por proposiciones at\'omicas, negaciones de otras f\'ormulas y conjunciones de dos f\'ormulas.\\\\
	Luego de codificar se tiene que transformar a su notaci\'on postfix o tambi\'en llamada \textit{notaci\'on polaca inversa} con la cual facilitar\'a la creaci\'on de un \textit{parse tree} para asignar valores \textbf{T} o \textbf{F} a cada nodo. Al tener el parse tree correspondiente a la f\'ormula que se eval\'ua asignamos \textbf{T} al nodo que encabeza el \'arbol. Esto implica asumir que la f\'ormula completa es verdadera y a partir de ello se puede extender esta asignaci\'on hacia los nodos hijos del \'arbol aplicando reglas sem\'anticas de los conectores l\'ogicos. \\\\
	Si el nodo principal es una conjunci\'on $\phi \land \psi$ entonces $\phi$ y $\psi$ deben ser verdaderas. Por el contrario, si el nodo es una negaci\'on $\neg \phi$ quiere decir que la subf\'ormula $\phi$ es falsa. Este procedimiento se aplica recursivamente hasta llegar a los nodos hoja, los cuales corresponden a \'atomos proposicionales. \\\\
	De esta forma se obtiene una asignaci\'on de valores de verdad que satisface la f\'ormula. En caso que las asignaciones conduzcan a una contradicci\'on (por ejemplo, se tiene $p \equiv \textbf{T}$ y $\neg p\equiv \textbf{T}$) se descarta el camino recorrido o incluso puede significar que la f\'ormula es \textit{insatisfacible}. \\\\
	Adicionalmente para una mayor eficiencia en espacio y tiempo, detectar y reutilizar \'atomos proposicionales podemos construir en cambio un DAG (Directed Acyclic Graph).
	\subsection{Ejemplo}
	Dada la siguiente f\'ormula:
	\[ ((p \rightarrow q) \wedge (\neg r \vee p)) \]
	El primer paso es aplicar la codificaci\'on mencionada anteriormente
	\begin{align*}
		\phi &= ((p \rightarrow q) \wedge (\neg r \vee p)) \\
		T(\phi) &= T(((p \rightarrow q) \wedge (\neg r \vee p)) ) \\
		&= T(p \rightarrow q) \wedge T(\neg r \vee p) \\
		&= \neg(T(p) \wedge \neg T(q)) \wedge \neg(\neg T(\neg r) \wedge \neg T(p)) \\
		T(\phi) &= \neg(p \wedge \neg q) \wedge \neg(\neg \neg r \wedge \neg p)
	\end{align*}
	El siguiente paso es transformar la f\'ormula codificada a su notaci\'on postfix, para esto hay que descomponer la f\'ormula en \textit{tokens} de la siguiente manera:
	\[ [\neg, (, p, \wedge, \neg, q, ), \wedge, \neg, (, \neg, \neg, r, \wedge, \neg, p, )] \]
	Para transformas a su notaci\'on postfix se tiene que saber que se eval\'uan los operadores seg\'un la precedencia, en donde la negaci\'on tiene la mayor precedencia por lo que se eval\'ua primero y desp\'ues la conjunci\'on.
	\begin{center}
		\begin{longtable}{|c|l|l|l|}
			\hline
			\textbf{Token} & \textbf{Acción} & \textbf{Salida} & \textbf{Stack} \\
			\hline
			\endfirsthead
			\hline
			\textbf{Token} & \textbf{Acción} & \textbf{Salida} & \textbf{Stack} \\
			\hline
			\endhead
			
			$\neg$ & Apilar operador & & $\neg$ \\
			\hline
			( & Apilar paréntesis & & $\neg$, ( \\
			\hline
			$p$ & Agregar a salida & $p$ & $\neg$, ( \\
			\hline
			$\wedge$ & Apilar operador & $p$ & $\neg$, (, $\wedge$ \\
			\hline
			$\neg$ & Apilar operador & $p$ & $\neg$, (, $\wedge$, $\neg$ \\
			\hline
			$q$ & Agregar a salida & $p\ q$ & $\neg$, (, $\wedge$, $\neg$ \\
			\hline
			) & Desapilar hasta ( & $p\ q\ \neg\ \wedge$ & $\neg$ \\
			\hline
			$\wedge$ & Apilar operador & $p\ q\ \neg\ \wedge$ & $\neg$, $\wedge$ \\
			\hline
			$\neg$ & Apilar operador & $p\ q\ \neg\ \wedge$ & $\neg$, $\wedge$, $\neg$ \\
			\hline
			( & Apilar paréntesis & $p\ q\ \neg\ \wedge$ & $\neg$, $\wedge$, $\neg$, ( \\
			\hline
			$\neg$ & Apilar operador & $p\ q\ \neg\ \wedge$ & $\neg$, $\wedge$, $\neg$, (, $\neg$ \\
			\hline
			$\neg$ & Apilar operador & $p\ q\ \neg\ \wedge$ & $\neg$, $\wedge$, $\neg$, (, $\neg$, $\neg$ \\
			\hline
			$r$ & Agregar a salida & $p\ q\ \neg\ \wedge\ r$ & $\neg$, $\wedge$, $\neg$, (, $\neg$, $\neg$ \\
			\hline
			$\wedge$ & Apilar operador & $p\ q\ \neg\ \wedge\ r$ & $\neg$, $\wedge$, $\neg$, (, $\neg$, $\neg$, $\wedge$ \\
			\hline
			$\neg$ & Apilar operador & $p\ q\ \neg\ \wedge\ r$ & $\neg$, $\wedge$, $\neg$, (, $\neg$, $\neg$, $\wedge$, $\neg$ \\
			\hline
			$p$ & Agregar a salida & $p\ q\ \neg\ \wedge\ r\ p$ & $\neg$, $\wedge$, $\neg$, (, $\neg$, $\neg$, $\wedge$, $\neg$ \\
			\hline
			) & Desapilar hasta ( & $p\ q\ \neg\ \wedge\ r\ p\ \neg\ \wedge \neg \neg$ & $\neg$, $\wedge$, $\neg$ \\
			\hline
			& Vaciar pila & $p\ q\ \neg\ \wedge\ r\ p\ \neg\ \wedge \neg\ \neg\ \neg\ \wedge\ \neg$ &  \\
			\hline
		\end{longtable}
	\end{center}
	\section{Algoritmo}
	
	\section{Implementaci\'on}
	
	\section{Conclusiones}
	
\end{document}



