%%%%%%%%%%%%%%%%%%%%%%%%%%%%%%%%%%%%%%%%%%%%%%%%%%%%%%%%%%%%%%%%%%%%%%%%%%%%%%%%
%
% Topico     : Informe Sat Lineal
%
% Autor      : Miguel Olivares Morales,Benjamín Riveros Landeros.
%
% Santiago de Chile, 19/05/2025
%
%%%%%%%%%%%%%%%%%%%%%%%%%%%%%%%%%%%%%%%%%%%%%%%%%%%%%%%%%%%%%%%%%%%%%%%%%%%%%%%%

\documentclass{report}

\usepackage{epsfig}
\usepackage{graphicx}
\usepackage{array}
\usepackage{geometry}
\usepackage{longtable}
\usepackage{listings}
\usepackage{amsmath}
\usepackage{amssymb}
\usepackage{tikz}
\usepackage{tikz-qtree}
\usepackage[linesnumbered]{algorithm2e}


\renewcommand*\thesection{\arabic{section}}

\newcommand \minitab{\hspace*{15 pt}}
\newcommand{\floor}[1]{\left\lfloor #1 \right\rfloor}

\hyphenation{ins-truc-ciones}

\lstdefinelanguage{Lex}{
	morekeywords={%%
		%{, %%, %%
			BEGIN, BEGIN, yytext, yylval, yywrap,
			printf, return, int, char, void
		},
		sensitive=true,
		morecomment=[l]{//},
		morecomment=[s]{/*}{*/},
		morestring=[b]",
	}
	
	\lstset{
		basicstyle=\ttfamily\small,
		keywordstyle=\color{blue}\bfseries,
		commentstyle=\color{gray},
		stringstyle=\color{red},
		numbers=left,
		numberstyle=\tiny,
		stepnumber=1,
		numbersep=5pt,
		tabsize=4,
		frame=single,
		breaklines=true,
		showstringspaces=false
	}

\begin{document}
	\RestyleAlgo{ruled}
	\begin{titlepage}
		\begin{center}
			\includegraphics[width=0.5\textwidth]{../Imagenes/LOGO-USACH_COLOR}
		\end{center}
		\begin{center}
			{\bf Departamento de Matem\'atica y Ciencia de la Computaci\'on}
		\end{center}
		\vspace{3cm}
		\begin{center}
			{\Large \bf Tarea 1 - SAT Lineal}
			~ \\ 
			~ \\ 
			~ \\
			~ \\
			\begin{tabular}{c c c}
				Miguel Olivares Morales & ~~~~~~~ & Benjam\'in Riveros Landeros \\
				miguel.olivares@usach.cl & & benjamin.riveros.l@usach.cl \\
			\end{tabular}
			~ \\ 
			~ \\ 
			~ \\
			~ \\
			\begin{tabular}{c c c}
				L\'ogica Computacional - 22625 & ~~~~~~~~~~~~~~~~~ & Semestre Oto\~no 2025 \\
				Licenciatura en Ciencia de la Computaci\'on & &  \\
			\end{tabular}
		\end{center}
	\end{titlepage}
	
	\section{Introducci\'on}
	El problema de determinar si las variables de una f\'ormula booleana pueden ser reemplazadas con valores \textbf{T} o \textbf{F} de tal forma que la f\'ormula de como resultado \textbf{T} se denomina problema de satisfacibilidad booleana o SAT. Si al evaluar la f\'ormula esta da como resultado \textbf{T}, entonces se dice que es satisfactoria.
	\section{Procedimiento}
	Las f\'ormulas que ser\'an analizadas primero tendr\'an que ser codificadas seg\'un la siguiente gramatica:
	\[ \phi ::= p \hspace{1mm}|\hspace{1mm} (\neg \phi)  \hspace{1mm}|\hspace{1mm} (\phi \land \phi) \]
	Para esto usamos el siguiente esquema de traducci\'on: \\\\
	\begin{minipage}[t]{0.45\textwidth}
		$ T(p)=p $ \\\\
		$ T(\phi_1 \land \phi_2)=T(\phi_1) \land T(\phi_2) $ \\\\
		$ T(\phi_1 \rightarrow \phi_2) = \neg(T(\phi_1) \land \neg T(\phi_2)) $
	\end{minipage}
	\hfill
	\begin{minipage}[t]{0.45\textwidth}
		$ T(\neg \phi) = \neg T(\phi) $ \\\\
		$ T(\phi_1 \lor \phi_2) = \neg(\neg T(\phi_1) \land \neg T(\phi_2)) $
	\end{minipage} \vspace*{5mm} \\ 
	Esto quiere decir que se analizar\'an f\'ormulas compuestas por proposiciones at\'omicas, negaciones de otras f\'ormulas y conjunciones de dos f\'ormulas.\\\\
	Luego de codificar se tiene que transformar a su notaci\'on postfix o tambi\'en llamada \textit{notaci\'on polaca inversa} con la cual facilitar\'a la creaci\'on de un \textit{parse tree} para asignar valores \textbf{T} o \textbf{F} a cada nodo. Al tener el parse tree correspondiente a la f\'ormula que se eval\'ua asignamos \textbf{T} al nodo que encabeza el \'arbol. Esto implica asumir que la f\'ormula completa es verdadera y a partir de ello se puede extender esta asignaci\'on hacia los nodos hijos del \'arbol aplicando reglas sem\'anticas de los conectores l\'ogicos. \\\\
	Si el nodo principal es una conjunci\'on $\phi \land \psi$ entonces $\phi$ y $\psi$ deben ser verdaderas. Por el contrario, si el nodo es una negaci\'on $\neg \phi$ quiere decir que la subf\'ormula $\phi$ es falsa. Este procedimiento se aplica recursivamente hasta llegar a los nodos hoja, los cuales corresponden a \'atomos proposicionales. \\\\
	De esta forma se obtiene una asignaci\'on de valores de verdad que satisface la f\'ormula. En caso que las asignaciones conduzcan a una contradicci\'on (por ejemplo, se tiene $p \equiv \textbf{T}$ y $\neg p\equiv \textbf{T}$) se descarta el camino recorrido o incluso puede significar que la f\'ormula es \textit{insatisfacible}. \\\\
	Adicionalmente para una mayor eficiencia en espacio y tiempo, detectar y reutilizar \'atomos proposicionales podemos construir en cambio un DAG (Directed Acyclic Graph).
	\subsection{Ejemplo}
	Dada la siguiente f\'ormula:
	\[ ((p \rightarrow q) \wedge (\neg r \vee p)) \]
	El primer paso es aplicar la codificaci\'on mencionada anteriormente
	\begin{align*}
		\phi &= ((p \rightarrow q) \wedge (\neg r \vee p)) \\
		T(\phi) &= T(((p \rightarrow q) \wedge (\neg r \vee p)) ) \\
		&= T(p \rightarrow q) \wedge T(\neg r \vee p) \\
		&= \neg(T(p) \wedge \neg T(q)) \wedge \neg(\neg T(\neg r) \wedge \neg T(p)) \\
		T(\phi) &= \neg(p \wedge \neg q) \wedge \neg(\neg \neg r \wedge \neg p)
	\end{align*}
	El siguiente paso es transformar la f\'ormula codificada a su notaci\'on postfix, para esto hay que descomponer la f\'ormula en \textit{tokens} de la siguiente manera:
	\[ [\neg, (, p, \wedge, \neg, q, ), \wedge, \neg, (, \neg, \neg, r, \wedge, \neg, p, )] \]
	Para transformar a su notaci\'on postfix se tiene que saber que se eval\'uan los operadores seg\'un la precedencia, en donde la negaci\'on tiene la mayor precedencia por lo que se eval\'ua primero luego de esta, la conjunci\'on.
	\begin{center}
		\begin{longtable}{|c|l|l|l|}
			\hline
			\textbf{Token} & \textbf{Acción} & \textbf{Salida} & \textbf{Stack} \\
			\hline
			\endfirsthead
			\hline
			\textbf{Token} & \textbf{Acción} & \textbf{Salida} & \textbf{Stack} \\
			\hline
			\endhead
			
			$\neg$ & Apilar operador & & $\neg$ \\
			\hline
			( & Apilar paréntesis & & $\neg$, ( \\
			\hline
			$p$ & Agregar a salida & $p$ & $\neg$, ( \\
			\hline
			$\wedge$ & Apilar operador & $p$ & $\neg$, (, $\wedge$ \\
			\hline
			$\neg$ & Apilar operador & $p$ & $\neg$, (, $\wedge$, $\neg$ \\
			\hline
			$q$ & Agregar a salida & $p\ q$ & $\neg$, (, $\wedge$, $\neg$ \\
			\hline
			) & Desapilar hasta ( & $p\ q\ \neg\ \wedge$ & $\neg$ \\
			\hline
			$\wedge$ & Apilar operador & $p\ q\ \neg\ \wedge$ & $\neg$, $\wedge$ \\
			\hline
			$\neg$ & Apilar operador & $p\ q\ \neg\ \wedge$ & $\neg$, $\wedge$, $\neg$ \\
			\hline
			( & Apilar paréntesis & $p\ q\ \neg\ \wedge$ & $\neg$, $\wedge$, $\neg$, ( \\
			\hline
			$\neg$ & Apilar operador & $p\ q\ \neg\ \wedge$ & $\neg$, $\wedge$, $\neg$, (, $\neg$ \\
			\hline
			$\neg$ & Apilar operador & $p\ q\ \neg\ \wedge$ & $\neg$, $\wedge$, $\neg$, (, $\neg$, $\neg$ \\
			\hline
			$r$ & Agregar a salida & $p\ q\ \neg\ \wedge\ r$ & $\neg$, $\wedge$, $\neg$, (, $\neg$, $\neg$ \\
			\hline
			$\wedge$ & Apilar operador & $p\ q\ \neg\ \wedge\ r$ & $\neg$, $\wedge$, $\neg$, (, $\neg$, $\neg$, $\wedge$ \\
			\hline
			$\neg$ & Apilar operador & $p\ q\ \neg\ \wedge\ r$ & $\neg$, $\wedge$, $\neg$, (, $\neg$, $\neg$, $\wedge$, $\neg$ \\
			\hline
			$p$ & Agregar a salida & $p\ q\ \neg\ \wedge\ r\ p$ & $\neg$, $\wedge$, $\neg$, (, $\neg$, $\neg$, $\wedge$, $\neg$ \\
			\hline
			) & Desapilar hasta ( & $p\ q\ \neg\ \wedge\ r\ p\ \neg\ \wedge \neg \neg$ & $\neg$, $\wedge$, $\neg$ \\
			\hline
			& Vaciar pila & $p\ q\ \neg\ \wedge\ r\ p\ \neg\ \wedge \neg\ \neg\ \neg\ \wedge\ \neg$ &  \\
			\hline
		\end{longtable}
	\end{center}
	Finalmente al tener el \textit{parse tree} se busca una asignaci\'on de valores que cumpla la satisfacibilidad de la f\'ormula \\
	\begin{center}
		\begin{tikzpicture}
			[
			level 1/.style={sibling distance=25mm},
			level 2/.style={sibling distance=15mm},
			]
			\node {$\wedge$}
			child {node {$\neg$}
				child{node {$\wedge$}
					child{node {$p$}}
					child{node {$\neg$}
						child{node {$q$}}}}
				}
			child {
				node {$\neg$}
				child{node {$\wedge$}
				child {node {$\neg$}
					child {node {$\neg$}
						child {node {$r$}}}}
				child {node {$\neg$}
					child {node {$p$}}}}
			};
		\end{tikzpicture} \\
		Y al aplicar algoritmo backtracking se tendría por ejemplo: \\\vspace*{5mm}
		\begin{tikzpicture}
			[
			level 1/.style={sibling distance=25mm},
			level 2/.style={sibling distance=15mm},
			]
			\node {$T$}
			child {node {$T$}
				child{node {$F$}
					child{node {$F$}}
					child{node {$T$}
						child{node {$F$}}}}
			}
			child {
				node {$T$}
				child{node {$F$}
					child {node {$F$}
						child {node {$T$}
							child {node {$F$}}}}
					child {node {$T$}
						child {node {$F$}}}}
			};
		\end{tikzpicture}
	\end{center}
	
	\newpage
	\section{Algoritmo}
	\subsection{Codificación de fórmula}
	Para simplificar el procesamiento, transformamos la fórmula booleana en una forma equivalente que usa solo negación y conjunción, eliminando implicaciones y disyunciones mediante equivalencias lógicas.
	A continuación se muestra algoritmo recursivo $T(\phi)$ aplicando estas reglas:
	\begin{algorithm}
		\caption{Algoritmo recursivo $T(\phi)$ para codificar una fórmula booleana $\phi$}
		\KwIn{$\phi$}
		\KwOut{$T(\phi)$}
		\If{$\phi$ es un átomo proposicional $p$}{
			\Return{$p$}
		}
		\If{$\phi$ es una negación ($\neg \phi$)}{
			\Return{$\neg T(\phi)$}
		}
		\If{$\phi$ es una conjunción ($\phi_1 \wedge \phi_2$)}{
			\Return{$T(\phi_1) \wedge T(\phi_2)$}
		}
		\If{$\phi$ es una disyunción ($\phi_1 \vee \phi_2$)}{
			\Return{$\neg(\neg T(\phi_1) \wedge \neg T(\phi_2))$}
		}
		\If{$\phi$ es una implicancia ($\phi_1 \rightarrow \phi_2$)}{
			\Return{$\neg(T(\phi_1) \wedge \neg T(\phi_2))$}
		}
		\If{$\phi$ tiene una fórmula no reconocida}{
			Reportar error: fórmula inválida			
		}
	\end{algorithm}
	\subsection{Conversión a Notación Postfija}
	Para transformar las fórmulas lógicas desde su forma infija a una notación postfija (también conocida como notación polaca inversa), se implementó un algoritmo inspirado en el \textit{Shunting Yard Algorithm} de Edsger Dijkstra. Esta notación facilita el análisis y la evaluación de fórmulas lógicas, ya que elimina la ambigüedad del orden de operaciones y permite un procesamiento más eficiente.
	\newpage
	\begin{algorithm}[hbt!]
		\caption{Conversión a notación postfija (inspirado en Shunting Yard Algorithm)}
		\KwIn{Fórmula codificada $T(\phi)$}
		\KwOut{Fórmula $T(\phi)$ en notación postfija}
		Inicializar una pila vacía \texttt{stack} \\
		Inicializar una cadena vacía \texttt{output} \\
		\For{cada token en la fórmula de entrada}{
			\If{el token es un operando}{
				Agregar token a \texttt{output}
			}
			\ElseIf{el token es un operador}{
				\While{pila no vacía y tope operador con mayor o igual precedencia}{
					Desapilar operador y agregarlo a \texttt{output}
				}
				Apilar el operador actual
			}
			\ElseIf{el token es un paréntesis de apertura}{
				Apilar el token
			}
			\ElseIf{el token es un paréntesis de cierre}{
				\While{tope de pila no es paréntesis de apertura}{
					Desapilar operador y agregarlo a \texttt{output}
				}
				\If{pila vacía}{
					Reportar error: paréntesis desbalanceados
				}
				\Else{
					Desapilar paréntesis de apertura
				}
			}
		}
		\While{pila no vacía}{
			\If{tope es paréntesis}{
				Reportar error: paréntesis desbalanceados
			}
			\Else{
				Desapilar operador y agregarlo a \texttt{output}
			}
		}
		\KwRet{} \texttt{output}
	\end{algorithm}
	\newpage
	\subsection{Backtracking}
	Para encontrar la satisfacibilidad de la f\'ormula se usa un algoritmo de b\'usqueda que explora un \'arbol de posibles soluciones probando distintos caminos hasta encontrar una respuesta v\'alida o se haya explorado todas las posibilidades. Si un camino lleva a contradicci\'on, el algoritmo "retrocede" a un paso anterior y prueba \textbf{T} si antes era \textbf{F} o viceversa. \\
	\begin{algorithm}[H]
		\caption{Backtracking para verificar si una fórmula CNF es satisfacible}
		\KwIn{Fórmula $T(\phi)$, asignación booleana $A[0 \dots n{-}1]$, número de variables $n$, índice actual $i$}
		\KwOut{\textbf{true} si existe una asignación que satisface $F$, \textbf{false} en otro caso}
			\If{$i = n$}{
				$c \gets T(\phi)$
				\While{$c \neq$ nulo}{
					\If{\textbf{not} is\_satisfied($c$, $A$, $n$)}{
						\KwRet \textbf{false}
					}
					$c \gets c.derecha$
				}
				\KwRet \textbf{true}
			}
			
			$A[i] \gets$ \textbf{true} \\
			\If{backtrack($T(\phi)$, $A$, $n$, $i{+}1$)}{
				\KwRet \textbf{true}
			}
			
			$A[i] \gets$ \textbf{false} \\
			\If{backtrack($T(\phi)$, $A$, $n$, $i{+}1$)}{
				\KwRet \textbf{true}
			}
			
			\KwRet \textbf{false}
	\end{algorithm}
	\subsection{Satisfacibilidad}
	Comprueba si una cl\'ausula espec\'ifica se cumple bajo una asignaci\'on de valores de verdad a los \'atomos. \\
	\begin{algorithm}[H]
		\caption{Resuelve el problema de satisfacibilidad (SAT) para una fórmula en CNF}
		\KwIn{Fórmula $T(\phi)$ en forma normal conjuntiva, número de variables $n$}
		\KwOut{\textbf{true} si $T(\phi)$ es satisfacible, \textbf{false} en otro caso}
			Crear arreglo booleano $A[0 \dots n{-}1]$ inicializado en \textbf{false} \\
			\If{no se puede reservar memoria para $A$}{
				Imprimir ``Error al asignar memoria'' \\
				\KwRet \textbf{false}
			}
			
			$result \gets$ backtrack($T(\phi)$, $A$, $n$, $0$) \\
			Liberar memoria de $A$ \\
			\KwRet $result$
	\end{algorithm}
	\newpage
	\section{Implementaci\'on}
	\subsection{Codificaci\'on de f\'ormula}

	La construcci\'on del \'arbol sint\'actico a partir de la f\'ormula en notaci\'on postfija se basa en el uso de una pila y las operaciones fundamentales \texttt{push} y \texttt{pop} para manejar los nodos.

	\begin{itemize}
		\item Las \textbf{variables} se convierten en nodos hoja y se insertan en la pila.
		\item Los \textbf{operadores unarios} como $\neg$ toman un operando de la pila y crean un nodo unario.
		\item Los \textbf{operadores binarios} como $\wedge$, $\vee$ e $\rightarrow$ extraen dos operandos de la pila, crean un nuevo nodo con la operaci\'on correspondiente y lo reinsertan.
		\item Las \textbf{transformaciones l\'ogicas} se aplican para eliminar operadores como $\rightarrow$ y $\vee$ mediante equivalencias:
		\begin{itemize}
			\item $A \rightarrow B \equiv \neg (A \wedge \neg B)$
			\item $A \vee B \equiv \neg (\neg A \wedge \neg B)$
		\end{itemize}
	\end{itemize}

	El siguiente fragmento ejemplifica el manejo de operadores y la manipulaci\'on de la pila para construir el \'arbol:

	\begin{lstlisting}[language=Lex, caption={Manejo de pila y reescritura lógica durante la construcción del árbol}]
	"\\rightarrow" {
		Node* right = pop();
		Node* left = pop();
		Node* neg_right = create_not(right);
		Node* and_node = create_op(AND, left, neg_right);
		push(create_not(and_node));
	}

	"\\neg" {
		Node* child = pop();
		push(create_not(child));
	}

	"\\wedge" {
		Node* right = pop();
		Node* left = pop();
		push(create_op(AND, left, right));
	}

	"\\vee" {
		Node* right = pop();
		Node* left = pop();
		Node* neg_left = create_not(left);
		Node* neg_right = create_not(right);
		Node* and_node = create_op(AND, neg_left, neg_right);
		push(create_not(and_node));
	}
	\end{lstlisting}

	\newpage
	\subsection{Notaci\'on Postfix}

	Esta transformación simplifica el análisis posterior de la fórmula.

	El código en C:
	\begin{itemize}
		\item Define operadores lógicos con su precedencia y asociatividad.
		\item Procesa la entrada carácter por carácter, reconociendo paréntesis, operadores y variables.
		\item Aplica el algoritmo de Shunting Yard usando una pila.
		\item Genera la expresión postfija y detecta errores de paréntesis.
	\end{itemize}

	Este módulo permite transformar expresiones como \verb|$((p \rightarrow q) \wedge (\neg r \vee p))$| en \verb|p q \rightarrow r \neg p \vee \wedge|.

	A continuación, se muestra el fragmento más importante del código:


	\begin{lstlisting}[language=C, caption={Fragmento clave de la conversión a notación postfija}]
	Operator operators[] = {
		{"\\neg", 3, 1},       // Mayor precedencia, asociativo a la derecha
		{"\\wedge", 2, 0},     // Conjuncion
		{"\\vee", 1, 0},       // Disyuncion
		{"\\rightarrow", 0, 0},// Implicacion
		{NULL, 0, 0}
	};

	char* convert_to_postfix(const char* input) {
		char output[MAX_OUTPUT] = "";
		const char* stack[MAX_STACK];
		int top = 0;

		// Procesamiento de tokens y aplicacion del algoritmo de Shunting Yard
		while (...) {
			// Manejo de operadores
			if (is_operator(token)) {
				while (top > 0 && is_operator(stack[top - 1]) &&
					((precedence(stack[top - 1]) > precedence(token)) ||
					(precedence(stack[top - 1]) == precedence(token) &&
					!is_right_associative(token)))) {
					strcat(output, stack[--top]);
					strcat(output, " ");
				}
				stack[top++] = strdup(token);
			}
			// Manejo de parentesis y operandos...
		}

		// Vaciar la pila al final
		while (top > 0) {
			strcat(output, stack[--top]);
			strcat(output, " ");
		}

		return strdup(output);
	}
	\end{lstlisting}
	\subsection{Construcción del Árbol Sintáctico}

	La representación de una fórmula lógica se realiza mediante estructuras de árbol donde cada nodo representa un operador o una variable.

	\begin{itemize}
		\item Las \textbf{variables proposicionales} como \texttt{P} o \texttt{Q} se almacenan como nodos hoja.
		\item Los \textbf{operadores binarios} (\texttt{AND}, \texttt{OR}, \texttt{IMPL}) generan nodos internos con dos hijos.
		\item El \textbf{operador unario} \texttt{NOT} crea un nodo con un solo hijo.
	\end{itemize}

	\begin{lstlisting}[language=C, caption={Creación de nodos para el árbol sintáctico}]
	Node* create_var(const char* name) {
		Node* node = (Node*)malloc(sizeof(Node));
		node->type = VAR;
		node->var_name = strdup(name);
		node->left = node->right = NULL;
		return node;
	}

	Node* create_op(NodeType type, Node* left, Node* right) {
		Node* node = (Node*) malloc(sizeof(Node));
		node->type = type;
		node->left = left;
		node->right = right;
		node->var_name = NULL;
		return node;
	}

	Node* create_not(Node* child) {
		Node* node = (Node*) malloc(sizeof(Node));
		node->type = NOT;
		node->left = child;
		node->right = NULL;
		node->var_name = NULL;
		return node;
	}
	\end{lstlisting}

	\subsection{Copia y Liberación de Árboles}

	Se incluyen funciones auxiliares para copiar un árbol lógico o liberar toda su memoria, evitando fugas.

	\begin{lstlisting}[language=C, caption={Copia y liberación de árboles lógicos}]
	Node* copy_tree(Node* node) {
		if (!node) return NULL;

		Node* new_node = (Node*)malloc(sizeof(Node));
		new_node->type = node->type;
		new_node->var_name = node->var_name ? strdup(node->var_name) : NULL;
		new_node->left = copy_tree(node->left);
		new_node->right = copy_tree(node->right);
		return new_node;
	}

	void free_tree(Node* node) {
		if (!node) return;

		free_tree(node->left);
		free_tree(node->right);

		if (node->var_name) free(node->var_name);
		free(node);
	}
	\end{lstlisting}

	\subsection{Conteo de Variables Proposicionales}

	Esta función identifica cuántas variables únicas (A--Z) aparecen en el árbol lógico. Usa un arreglo booleano como bandera para marcar la presencia de cada letra.

	\begin{lstlisting}[language=C, caption={Conteo de variables proposicionales}]
	static void mark_vars(Node* node, bool vars[26]) {
		if (!node) return;

		if (node->type == VAR && node->var_name) {
			char c = node->var_name[0];
			if (c >= 'A' && c <= 'Z')
				vars[c - 'A'] = true;
		}

		mark_vars(node->left, vars);
		mark_vars(node->right, vars);
	}

	int get_num_vars(Node* node) {
		bool vars[26] = {false};
		mark_vars(node, vars);

		int count = 0;
		for (int i = 0; i < 26; i++)
			if (vars[i]) count++;

		return count;
	}
	\end{lstlisting}

	\subsection{Evaluación de Fórmulas Lógicas}

	Se puede evaluar una fórmula con una asignación booleana específica. Las variables se asocian a valores mediante su posición (A=0, B=1, ..., Z=25).

	\begin{lstlisting}[language=C, caption={Evaluación de una fórmula lógica}]
	bool eval_formula(Node* node, bool assignment[26]) {
		if (!node) return false;

		switch (node->type) {
			case VAR:
				return assignment[node->var_name[0] - 'A'];
			case NOT:
				return !eval_formula(node->left, assignment);
			case AND:
				return eval_formula(node->left, assignment) &&
					eval_formula(node->right, assignment);
			case OR:
				return eval_formula(node->left, assignment) ||
					eval_formula(node->right, assignment);
			default:
				return false;
		}
	}
	\end{lstlisting}

	\subsection{Verificación de Satisfacibilidad (SAT)}

	Se utiliza backtracking para verificar si existe alguna combinación de valores de verdad que satisfaga la fórmula lógica.

	\begin{lstlisting}[language=C, caption={Backtracking para resolver SAT}]
	bool solve_rec(Node* node, bool assignment[26], int idx, int max_vars) {
		if (idx == max_vars)
			return eval_formula(node, assignment);

		assignment[idx] = false;
		if (solve_rec(node, assignment, idx + 1, max_vars)) return true;

		assignment[idx] = true;
		if (solve_rec(node, assignment, idx + 1, max_vars)) return true;

		return false;
	}

	bool solve_sat(Node* formula, int num_vars) {
		bool assignment[26] = {false};
		return solve_rec(formula, assignment, 0, num_vars);
	}
	\end{lstlisting}

	\subsection{Evaluación de Cláusulas bajo una Asignación}

	Para verificar si una cláusula está satisfecha con una asignación específica de valores booleanos a las variables, se utiliza la función \texttt{is\_satisfied}. Esta función analiza un nodo que representa una cláusula (puede ser un literal o su negación) y devuelve \texttt{true} si la cláusula es verdadera con la asignación dada.

	\begin{itemize}
		\item Si el nodo es una variable, simplemente se consulta su valor en el arreglo de asignación.
		\item Si el nodo es una negación, se niega el valor asignado a la variable correspondiente.
	\end{itemize}

	\begin{lstlisting}[language=C, caption={Función para evaluar si una cláusula está satisfecha}]
	bool is_satisfied(Node* clause, bool* assignment, int num_vars){
		if (!clause) return false;

		if (clause->type == VAR){
			int var_index = atoi(clause->var_name);
			return assignment[var_index];
		} 
		else if (clause->type == NOT){
			return !assignment[atoi(clause->left->var_name)];
		}
		return false;
	}
	\end{lstlisting}

	\subsection{Algoritmo de Backtracking para SAT}

	El algoritmo de backtracking recorre todas las posibles asignaciones de valores de verdad para las variables, utilizando recursión:

	\begin{itemize}
		\item Cuando se asignan valores a todas las variables, se verifica si todas las cláusulas de la fórmula están satisfechas.
		\item Si se encuentra una asignación que satisface todas las cláusulas, se devuelve \texttt{true}.
		\item Si ninguna asignación es válida, se devuelve \texttt{false}.
	\end{itemize}

	La función \texttt{solve\_sat} es la interfaz principal que inicializa la asignación y llama a la función recursiva \texttt{backtrack}.

	\begin{lstlisting}[language=C, caption={Backtracking para resolver SAT}]
	bool backtrack(Node* formula, bool* assignment, int num_vars, int index){
		if (index == num_vars){
			Node* current_clause = formula;
			while (current_clause){
				if (!is_satisfied(current_clause, assignment, num_vars)){
					return false;
				}
				current_clause = current_clause->right;
			}
			return true;
		}

		assignment[index] = true;
		if (backtrack(formula, assignment, num_vars, index + 1)) return true;

		assignment[index] = false;
		if (backtrack(formula, assignment, num_vars, index + 1)) return true;

		return false;
	}

	bool solve_sat(Node* formula, int num_vars){
		bool* assignment = malloc(num_vars * sizeof(bool));
		if (!assignment) {
			fprintf(stderr, "Error al asignar memoria.\n");
			return false;
		}

		for (int i = 0; i < num_vars; i++){
			assignment[i] = false;
		}

		bool result = backtrack(formula, assignment, num_vars, 0);

		free(assignment);
		return result;
	}
	\end{lstlisting}
	\newpage
	\section{Conclusiones}
	Hemos presentado la implementaci\'on de un algoritmo que determina si las variables de una f\'ormula booleana pueden ser reemplazadas con \textbf{T}	o \textbf{F} de modo que la f\'ormula entregue resultado \textbf{T}, es decir si es satisfacible o no. La tarea fundamental es codificar una f\'ormula de l\'ogica proposicional escrita en \LaTeX \;de tal forma que los \'unicos operadores que se eval\'uen sean la negaci\'on y conjunci\'on lo cual facilita la asignaci\'on de valores de verdadero o falso usando un \'arbol sint\'actico.
\end{document}



